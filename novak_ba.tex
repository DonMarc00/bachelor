%%% LaTeX-Vorlage Version 1.8 %%%

% Grundlegende Dokumenteneigenschaften gemäß DHBW-Vorgaben
\documentclass[a4paper,fontsize=11pt,oneside,parskip=half,headings=normal,listof=nochaptergap]{scrreprt} 
% \usepackage{showframe} % nur für Kontrolle der Ränder 

%%% Präambel einbinden (mit Festlegungen gemäß DHBW-Vorgaben) %%%
%%% Präambel %%%
% hier sollten keine Änderungen erforderlich sein
%
\usepackage[utf8]{inputenc}   % Zeichencodierung UTF-8 für Eingabe-Dateien
\usepackage[T1]{fontenc}      % Darstellung von Umlauten im PDF

\usepackage{listings}         % für Einbindung von Code-Listings
\usepackage{xcolor}
\usepackage{booktabs}

\renewcommand\lstlistlistingname{Listingsverzeichnis}

\definecolor{codegreen}{rgb}{0,0.6,0}
\definecolor{codegray}{rgb}{0.5,0.5,0.5}
\definecolor{codepurple}{rgb}{0.58,0,0.82}
\definecolor{backcolour}{rgb}{0.95,0.95,0.92}

\lstdefinestyle{mystyle}{
  backgroundcolor=\color{backcolour},
  commentstyle=\color{codegreen},
  keywordstyle=\color{magenta},
  numberstyle=\tiny\color{codegray},
  stringstyle=\color{codepurple},
  basicstyle=\ttfamily\footnotesize,
  breakatwhitespace=false,
  breaklines=true,
  captionpos=b,
  keepspaces=true,
  numbers=left,
  numbersep=5pt,
  showspaces=false,
  showstringspaces=false,
  showtabs=false,
  tabsize=2,
  numberbychapter=false
}

\lstset{style=mystyle}

\lstset{literate=             % erlaubt Sonderzeichen in Code-Listings 
  {Ö}{{\"O}}1
{Ä}{{\"A}}1
{Ü}{{\"U}}1
{ß}{{\ss}}2
{ü}{{\"u}}1
{ä}{{\"a}}1
{ö}{{\"o}}1
{€}{{\euro}}1
}

\usepackage[
  inner=35mm,outer=15mm,top=25mm,
  bottom=20mm,foot=12mm,includefoot
]{geometry}                 % Einstellungen für Ränder
\usepackage{tabularx}
\usepackage[ngerman]{babel} % Spracheinstellungen Deutsch
\usepackage[babel,german=quotes]{csquotes} % deutsche Anf.zeichen
\usepackage{enumerate}      % anpassbare Nummerier./Aufz.
\usepackage{graphicx}       % Einbinden von Grafiken
\usepackage[onehalfspacing]{setspace} % anderthalbzeilig

\usepackage{blindtext}      % Textgenerierung für Testzwecke
\usepackage{color}          % Verwendung von Farbe 
\usepackage{pgfplots}      % für Diagramme
\pgfplotsset{compat=1.16}
\usepackage{tikz}
\usepackage{graphicx}
\usepackage{listings}
\usepackage{xcolor}
\definecolor{delim}{RGB}{20,105,176}
\definecolor{numb}{RGB}{106, 109, 32}
\definecolor{string}{rgb}{0.64,0.08,0.08}
\definecolor{backcolour}{rgb}{1,1,1}

\lstdefinelanguage{json}{
    basicstyle=\ttfamily\small\color{black},
    backgroundcolor=\color{backcolour},
    numbers=left,
    numberstyle=\small\color{gray},
    stepnumber=1,
    numbersep=8pt,
    showstringspaces=false,
    breaklines=true,
    frame=single,
    literate=
     *{0}{{{\color{numb}0}}}{1}
      {1}{{{\color{numb}1}}}{1}
      {2}{{{\color{numb}2}}}{1}
      {3}{{{\color{numb}3}}}{1}
      {4}{{{\color{numb}4}}}{1}
      {5}{{{\color{numb}5}}}{1}
      {6}{{{\color{numb}6}}}{1}
      {7}{{{\color{numb}7}}}{1}
      {8}{{{\color{numb}8}}}{1}
      {9}{{{\color{numb}9}}}{1}
      {\{}{{{\color{delim}{\{}}}}{1}
      {\}}{{{\color{delim}{\}}}}}{1}
      {[}{{{\color{delim}{[}}}}{1}
      {]}{{{\color{delim}{]}}}}{1}
      {,}{{{\color{delim}{,}}}}{1}
      {:}{{{\color{delim}{:}}}}{1},
    morestring=[b]",
    stringstyle=\color{red},   % Dies färbt alle Strings rot.
}
\usepackage[printonlyused]{acronym}        % für ein Abkürzungsverzeichnis

\usepackage[                % Biblatex
  backend=biber,
  bibstyle=_dhbw_authoryear,maxbibnames=99,
  citestyle=authoryear,
  uniquename=true, useprefix=true,
  bibencoding=utf8]{biblatex}
%kein Punkt am Ende bei \footcite
%http://www.golatex.de/footcite-ohne-punkt-am-schluss-t4865.html
\renewcommand{\bibfootnotewrapper}[1]{\bibsentence#1}


%Reihenfolge der Autorennamen
%   
% http://golatex.de/viewtopic,p,80448.html#80448
% Argumente: siehe http://texwelt.de/blog/modifizieren-eines-biblatex-stils/
\DeclareNameFormat{sortname}{% Bibliographie
  \ifnum\value{uniquename}=0 % Normalfall
  \ifuseprefix%
    {%
    \usebibmacro{name:family-given}
    {\namepartfamily}
    {\namepartgiveni}
    {\namepartprefix}
    {\namepartsuffixi}%
    }
    {%
    \usebibmacro{name:family-given}
    {\namepartfamily}
    {\namepartgiveni}
    {\namepartprefixi}
    {\namepartsuffixi}%
    }%
  \fi
  \ifnum\value{uniquename}=1% falls nicht eindeutig, abgek. Vorname 
    {%
    \usebibmacro{name:family-given}
    {\namepartfamily}
    {\namepartgiveni}
    {\namepartprefix}
    {\namepartsuffix}%
    }%
  \fi
  \ifnum\value{uniquename}=2% falls nicht eindeutig, ganzer Vorname 
    {%
    \usebibmacro{name:family-given}
    {\namepartfamily}
    {\namepartgiven}
    {\namepartprefix}
    {\namepartsuffix}%
    }%
  \fi
  \usebibmacro{name:andothers}}

\DeclareNameFormat{labelname}{% für Zitate
  \ifnum\value{uniquename}=0 % Normalfall
  \ifuseprefix%
    {%
    \usebibmacro{name:family-given}
    {\namepartfamily}
    {\empty}
    {\namepartprefix}
    {\namepartsuffixi}%
    }
    {%
    \usebibmacro{name:family-given}
    {\namepartfamily}
    {\empty}
    {\namepartprefixi}
    {\namepartsuffixi}%
    }%
  \fi
  \ifnum\value{uniquename}=1% falls nicht eindeutig, abgek. Vorname 
    {%
    \usebibmacro{name:family-given}
    {\namepartfamily}
    {\namepartgiveni}
    {\namepartprefix}
    {\namepartsuffix}%
    }%
  \fi
  \ifnum\value{uniquename}=2% falls nicht eindeutig, ganzer Vorname 
    {%
    \usebibmacro{name:family-given}
    {\namepartfamily}
    {\namepartgiven}
    {\namepartprefix}
    {\namepartsuffix}%
    }%
  \fi
  \usebibmacro{name:andothers}}


\DeclareFieldFormat{extrayear}{% = the 'a' in 'Jones 1995a'
  \iffieldnums{labelyear}
  {\mknumalph{#1}}
  {\mknumalph{#1}}}

\renewcommand*{\multinamedelim}{\addslash}
\renewcommand*{\finalnamedelim}{\addslash}
\renewcommand*{\multilistdelim}{\addslash}
\renewcommand*{\finallistdelim}{\addslash}

\renewcommand{\nameyeardelim}{~}

% Literaturverzeichnis: Doppelpunkt zwischen Name (Jahr): Rest 
% http://de.comp.text.tex.narkive.com/Tn1HUIXB/biblatex-authoryear-und-doppelpunkt
\renewcommand{\labelnamepunct}{\addcolon\addspace}

% damit die Darstellung für Vollzitate von Primärquellen in 
% Fußnoten später auf "nicht fett" geändert werden kann 
% (nur für Zitate von Sekundärliteratur relevant)
\newcommand{\textfett}[1]{\textbf{#1}}

% für Zitate von Sekundärliteratur:
\newcommand{\footcitePrimaerSekundaer}[4]{%
  \renewcommand{\textfett}[1]{##1}%
  \footnote{\fullcite[#2]{#1}, zitiert nach \cite[#4]{#3}}%  
  \renewcommand{\textfett}[1]{\textbf{##1}}%
}

% Im Literaturverzeichnis: Autor (Jahr) fett
\renewbibmacro*{author}{%
  \ifboolexpr{%
    test \ifuseauthor%
    and
    not test {\ifnameundef{author}}
  }
  {\usebibmacro{bbx:dashcheck}
    {\bibnamedash}
    {\usebibmacro{bbx:savehash}%
      \textfett{\printnames{author}}%
      \iffieldundef{authortype}
      {\setunit{\addspace}}
      {\setunit{\addcomma\space}}}%
    \iffieldundef{authortype}
    {}
    {\usebibmacro{authorstrg}%
      \setunit{\addspace}}}%
  {\global\undef\bbx@lasthash
    \usebibmacro{labeltitle}%
    \setunit*{\addspace}}%
  \textfett{\usebibmacro{date+extrayear}}}

% Sonderfall: Quelle ohne Autor, aber mit Herausgeber
% Name des Herausgebers wird fett gedruckt
\renewbibmacro*{bbx:editor}[1]{%
  \ifboolexpr{%
    test \ifuseeditor%
    and
    not test {\ifnameundef{editor}}
  }
  {\usebibmacro{bbx:dashcheck}
    {\bibnamedash}
    {\textfett{\printnames{editor}}%
      \setunit{\addcomma\space}%
      \usebibmacro{bbx:savehash}}%
    \usebibmacro{#1}%
    \clearname{editor}%
    \setunit{\addspace}}%
  {\global\undef\bbx@lasthash
    \usebibmacro{labeltitle}%
    \setunit*{\addspace}}%
  \textfett{\usebibmacro{date+extrayear}}}

% Anpassungen für deutsche Sprache
\DefineBibliographyStrings{ngerman}{%
  nodate = {{o.J.}},
  urlseen = {{Abruf:}},
  ibidem = {{ebenda}}
}

% keine Anführungszeichen beim Titel im Literaturverzeichnis
\DeclareFieldFormat[article,book,inbook,inproceedings,manual,misc,phdthesis,thesis,online,report]{title}{#1\isdot}

\newcommand{\literaturverzeichnis}{%
  % nur Literaturverzeichnis
  % (als eigenes Kapitel)
  \phantomsection
  \addcontentsline{toc}{chapter}{Literaturverzeichnis}
  \spezialkopfzeile{Literaturverzeichnis}
  \defbibheading{lit}{\chapter*{Literaturverzeichnis}}
  \label{chapter:quellen}
  \printbibliography[heading=lit,notkeyword=ausblenden]
} % mit DHBW-spezifischen Einstellungen

\usepackage[hidelinks]{hyperref}       % URL-Formatierung, klickbare Verweise

\usepackage{tocloft}        % für Verzeichnis der Anhänge


\newcounter{anhcnt}
\setcounter{anhcnt}{0}
\newlistof{anhang}{app}{}

\newcommand{\anhang}[1]{%
  \refstepcounter{anhcnt}
  \setcounter{anhteilcnt}{0}
  \section*{Anhang \theanhcnt: #1}
  \addcontentsline{app}{section}{\protect\numberline{Anhang \theanhcnt}#1}\par
}

\newcounter{anhteilcnt}
\setcounter{anhteilcnt}{0}

\newcommand{\anhangteil}[1]{%
  \refstepcounter{anhteilcnt}
  \subsection*{Anhang~\arabic{anhcnt}/\arabic{anhteilcnt}: #1}
  \addcontentsline{app}{subsection}{\protect\numberline{Anhang \theanhcnt/\arabic{anhteilcnt}}#1}\par
}

\renewcommand{\theanhteilcnt}{Anhang \theanhcnt/\arabic{anhteilcnt}}

% vgl. S. 4 Paket-Beschreibung tocloft 	
% Einrückungen für Anhangverzeichnis
\makeatletter
\newcommand{\abstaendeanhangverzeichnis}{
  \renewcommand*{\l@section}{\@dottedtocline{1}{0em}{5.5em}}
  \renewcommand*{\l@subsection}{\@dottedtocline{2}{2.3em}{6.5em}}
}
\makeatother

% Abbildungs- und Tabellenverzeichnis
% Bezeichnungen
\renewcaptionname{ngerman}{\figurename}{Abb.}
\renewcaptionname{ngerman}{\tablename}{Tab.}
% Einrückungen
\makeatletter
\renewcommand*{\l@figure}{\@dottedtocline{1}{0em}{2.3em}}
\renewcommand*{\l@table}{\@dottedtocline{1}{0em}{2.3em}}
\makeatother


\usepackage{chngcntr}                % fortlaufende Zähler für Fußnoten, Abbildungen und Tabellen
\counterwithout{figure}{chapter}
\counterwithout{table}{chapter}
\counterwithout{footnote}{chapter}

\usepackage[automark]{scrlayer-scrpage}
%% Definitionen für Kopf- und Fußzeile auf normalen Seiten
\defpagestyle{kopfzeile}
{% Kopfdefinition
  (\textwidth,0pt)    % Länge der oberen Linie,Dicke der oberen Linie       
  {} % Definition für linke Seiten im doppelseitigen Layout
    {} % Definition für rechte Seiten im doppelseitigen Layout      
    {  % Definition für Seiten im einseitigen Layout
      \makebox[0pt][l]{\rightmark}% 
      \makebox[\linewidth]{}% 
    }
  (\textwidth, 0.4pt) % Untere Linienlänge, Untere Liniendicke
}
{% Fußdefinition
  (\textwidth,0pt)    % Obere Linienlänge, Obere Liniendicke
  {} % Definition für linke Seiten im doppelseitigen Layout
    {} % Definition für rechte Seiten im doppelseitigen Layout
    {  % Definition für Seiten im einseitigen Layout
      \makebox[\linewidth]{}%
      \makebox[0pt][r]{\pagemark}%
    }
  (\textwidth, 0pt)   % Länge der unteren Linie,Dicke der unteren Linie
}

%% Definitionen für Kopf- und Fußzeile auf ersten Seiten eines Kapitels
\defpagestyle{kapitelkopfzeile}
{% Kopfdefinition
  (\textwidth,0pt)    % Länge der oberen Linie,Dicke der oberen Linie       
  {} % Definition für linke Seiten im doppelseitigen Layout
    {} % Definition für rechte Seiten im doppelseitigen Layout      
    {}  % Definition für Seiten im einseitigen Layout
  (\textwidth, 0pt) % Untere Linienlänge, Untere Liniendicke
}
{% Fußdefinition
  (\textwidth,0pt)    % Obere Linienlänge, Obere Liniendicke
  {} % Definition für linke Seiten im doppelseitigen Layout
    {} % Definition für rechte Seiten im doppelseitigen Layout
    {  % Definition für Seiten im einseitigen Layout
      \makebox[\linewidth]{}%
      \makebox[0pt][r]{\pagemark}%
    }
  (\textwidth, 0pt)   % Länge der unteren Linie,Dicke der unteren Linie
}

%% Definitionen für Kopf- und Fußzeile im Anhang und bei Quellenverzeichnisse
\newcommand{\spezialkopfzeileBezeichnung}{}
\defpagestyle{spezialkopfzeile}
{% Kopfdefinition
  (\textwidth,0pt)    % Länge der oberen Linie,Dicke der oberen Linie       
  {} % Definition für linke Seiten im doppelseitigen Layout
    {} % Definition für rechte Seiten im doppelseitigen Layout      
    {  % Definition für Seiten im einseitigen Layout
      \makebox[0pt][l]{\spezialkopfzeileBezeichnung}% 
      \makebox[\linewidth]{}% 
    }
  (\textwidth, 0.4pt) % Untere Linienlänge, Untere Liniendicke
}
{% Fußdefinition
  (\textwidth,0pt)    % Obere Linienlänge, Obere Liniendicke
  {} % Definition für linke Seiten im doppelseitigen Layout
    {} % Definition für rechte Seiten im doppelseitigen Layout
    {  % Definition für Seiten im einseitigen Layout
      \makebox[\linewidth]{}%
      \makebox[0pt][r]{\pagemark}%
    }
  (\textwidth, 0pt)   % Länge der unteren Linie,Dicke der unteren Linie
}

\newcommand\spezialkopfzeile[1]{%
  \renewcommand\spezialkopfzeileBezeichnung{#1}
  \pagestyle{spezialkopfzeile}
}

% Standard-Pagestyle auswählen
\pagestyle{kopfzeile}

% keine Kopfzeile anzeigen auf Seiten, auf denen ein 
% Kapitel beginnt oder das Inhalts-/Abbildungs-/Tabellenverzeichnis steht 
\renewcommand{\chapterpagestyle}{kapitelkopfzeile}
\tocloftpagestyle{kapitelkopfzeile}

		 % für schöne Kopfzeilen 

\usepackage{textcomp}            % erlaubt EUR-Zeichen in Eingabedatei
\usepackage{eurosym}             % offizielles EUR-Symbol in Ausgabe
\renewcommand{\texteuro}{\euro}  % ACHTUNG: nach hyperref aufrufen!

\usepackage{scrhack}             % stellt Kompatibilität zw. KOMA-Script
% (scrreprt) und anderen Paketen her

% Anpassung der Abstände bei Kapitelüberschriften
% (betrifft auch Inhalts-, Abbildungs- und Tabellenverzeichnis)
\renewcommand*\chapterheadstartvskip{\vspace*{-\topskip}}
\newcommand{\myBeforeTitleSkip}{1mm}
\newcommand{\myAfterTitleSkip}{10mm}
\setlength\cftbeforetoctitleskip{\myBeforeTitleSkip}
\setlength\cftbeforeloftitleskip{\myBeforeTitleSkip}
\setlength\cftbeforelottitleskip{\myBeforeTitleSkip}

\setlength\cftaftertoctitleskip{\myAfterTitleSkip}
\setlength\cftafterloftitleskip{\myAfterTitleSkip}
\setlength\cftafterlottitleskip{\myAfterTitleSkip}
%%% Ende der Präambel %%%


%%% Name der eigenen Literatur-Datenbank (ggf. anpassen) %%%
\bibliography{includes/literatur.bib}

\begin{document}
%%% Deckblatt einbinden %%% 
% Anpassungen nötig (Name, Titel etc.)
% HIER EDITIEREN: 
% Typ der Arbeit (für Deckblatt und Metadaten)
% - bitte Zutreffendes auswählen
%\newcommand{\typMeinerArbeit}{1. Projektarbeit}
%\newcommand{\typMeinerArbeit}{2. Projektarbeit}
%\newcommand{\typMeinerArbeit}{Seminararbeit}
\newcommand{\typMeinerArbeit}{Bachelorarbeit}

% Thema der Arbeit (für ehrenwörtliche Erklärung, ohne Umbrüche)
% HIER EDITIEREN: 
\newcommand{\themaMeinerArbeit}{Document Understanding Transformers in der Dokumentenverarbeitung}

% Vorname, Name der Autorin/des Autors (für Titelseite und Metadaten)
% HIER EDITIEREN:
\newcommand{\meinName}{Marc Novak}

\thispagestyle{empty}

\begin{spacing}{1}
  \begin{center}
    ~\vspace{0mm}

    % HIER EDITIEREN: Titel der Arbeit
    {\sffamily
      \LARGE
      % \Large  % bei sehr langen Titeln ggf. etwas kleinere Schriftart wählen
      \textbf{Document Understanding Transformers in der Dokumentenverarbeitung}

      \bigskip
      \textbf{}
    }


    \vspace{15mm}

    % Typ wird automatisch eingefügt (oben festlegen)
    {\Large \typMeinerArbeit}

    \vspace{1cm}

    % HIER ggf. EDITIEREN
    vorgelegt am \today

    \vspace{15mm}

    Fakultät Wirtschaft
    \medskip

    Studiengang Wirtschaftsinformatik
    \medskip

    % HIER EDITIEREN: Kurs eintragen
    Kurs WI2021G

    \vspace{10mm}

    von

    \vspace{10mm}

    % Vorname und Name wird automatisch eingefügt (oben festlegen) 
    {\large\textsc{\meinName}}

    \vspace{10mm}
  \end{center}

  \vfill

  % HIER EDITIEREN: Name des Unternehmens, Name der Betreuerin/des Betreuers
  \begin{tabular}{ll}
    Betreuer in der Ausbildungsstätte:        & DHBW Stuttgart:                                                    \\
    \hspace{0.4\linewidth}                    &                                                                    \\
    ELO Digital Office GmbH & $\langle$ Titel, Vorname und Nachname $\rangle$                    \\
    Dr. Konstantin Hauch
                                              & $\langle$ der/des wissenschaftlichen Betreuerin/Prüferin $\rangle$ \\
    Teamleiter Team KI                                                     \\
    \\
    Unterschrift der Betreuerin/des Betreuers                                                                      \\
  \end{tabular}


  \vspace{1cm}
  %(etwas Platz für die Unterschrift der Betreuerin/des Betreuers aus der Ausbildungsstätte)
\end{spacing}

% falls ein Vertraulichkeitsvermerk erforderlich ist,
% die Kommentarzeichen in den nachfolgenden Zeilen entfernen:

%\begin{center}
%\small
%\textbf{Vertraulichkeitsvermerk}:
%Der Inhalt dieser Arbeit darf weder als Ganzes noch in Auszügen \\
%Personen außerhalb des Prüfungs- und Evaluationsverfahrens zugänglich gemacht werden, sofern keine anders lautende Genehmigung des Dualen Partners vorliegt. 
%\end{center}

% Meta-Daten für PDF-Datei basierend auf obigen Angaben
\hypersetup{pdftitle={\themaMeinerArbeit}}
\hypersetup{pdfauthor={\meinName}}
\hypersetup{pdfsubject={\typMeinerArbeit\ DHBW Stuttgart \the\year}}


%%% Umstellung der Seiten-Nummerierung auf i, ii, iii ... %%%
\pagenumbering{Roman}

%%% Inhalts-, Abbildungs-, Tabellenverzeichnisse %%%
% sollen einzeilig gesetzt werden, um Platz zu sparen 
\begin{spacing}{1}
  \tableofcontents
  \clearpage
  \chapter*{Abkürzungsverzeichnis}
\addcontentsline{toc}{chapter}{Abkürzungsverzeichnis}

\begin{acronym}[DHBW]
  % Argument definiert die Breite der ersten Spalte anhand des längsten vorkommenden Eintrags
  \acro{CRM}{Customer Relationship Management}
  \acro{DHBW}{Duale Hochschule Baden-Württemberg}
  \acro{IEEE}{Institute of Electrical and Electronics Engineers}
  \acro{ITIL}{IT Infrastructure Library}
  \acro{RoI}{Return-On-Invest}
  \acro{UCS}{Universal Character Set}
  \acro{UTF-8}{8-Bit UCS Transformation Format}
\end{acronym}

\vspace{2em}


  \clearpage
  \thispagestyle{kapitelkopfzeile}
  \listoffigures
  \phantomsection
  \addcontentsline{toc}{chapter}{Abbildungsverzeichnis} % Abb.verz. ins Inh.verz. aufnehmen

  \clearpage
  \listoftables
  \phantomsection
  \addcontentsline{toc}{chapter}{Tabellenverzeichnis}   % Tab.verz. ins Inh.verz. aufnehmen
  \clearpage
  \lstlistoflistings
  \addcontentsline{toc}{chapter}{Listingsverzeichnis}   % Lst.verz. ins Inh.verz. aufnehmen
\end{spacing}

%%% Umstellung der Seiten-Nummerierung auf 1, 2, 3 ... %%%
\cleardoublepage
\pagenumbering{arabic}

%%% Ihr eigentlicher Inhalt %%%
% Empfehlung: strukturieren Sie Ihren Text in einzelnen Dateien 
% und binden Sie diese hier mit \input{includes/dateiname.tex} ein
\chapter{Einleitung}

In den vergrangenen Jahren hat \ac{KI} in Unternehmen zunehmend an Bedeutung gewonnen. Seit 2019 verzeichnet der KI-Software Markt einen hohes Wachstum. Es wird davon ausgegangen, dass dieses Wachstum bis 2025 mit über 26 \% pro Jahr anhalten wird. \footcites[Vgl.][]{howarth_57_2024} Die Anwendungsbereiche von KI-Systemen sind vielfältig und reichen von der Automatisierung von Prozessen, über die Analyse von großen Datenmengen bis hin zur Vorhersage von zukünftigen Ereignissen.

Einer der beliebtesten Anwendungsbereiche von KI-Systemen ist die Dokumentenverarbeitung. Eine Befragung von 1420 IT-Fachkräften ergab, dass 28 \% der zugehörigen Unternehmen KI-Systeme zur Dokumentenverarbeitung einsetzen (s. Abb. \ref{fig:ai_tech_distribution}). \footcites[Vgl.][]{rackspace_most_2023} Die Dokumentenverarbeitung umfasst die Extraktion von Informationen, die Klassifizierung und die automatisierte Verarbeitung von Dokumenten.\footcites[Vgl.][S.1]{esposito_intelligent_2005} Die Verarbeitung von Dokumenten ist in vielen Unternehmen ein zeitaufwändiger Prozess, der durch den Einsatz von KI-Systemen automatisiert und beschleunigt werden kann.\footcites[Vgl.][S.11]{dutt_now_2024}

\pgfplotsset{compat=1.17} % Use the version of pgfplots you have installed

\begin{figure}[htb]
    \centering
\begin{tikzpicture}
    \begin{axis}[
        xbar, % Horizontal bars
        xmin=0, xmax=60, % Set the minimum and maximum x-coordinates
        width=11cm, height=9cm, % Width and height of the plot
        enlarge y limits=0.1, % Add some space between bars
        xlabel={Share of respondents in \%}, % Label for the x-axis
        symbolic y coords={
            Speech recognition,
            Image recognition,
            Facial recognition,
            Sales and marketing analytics,
            Document processing,
            Intelligent search,
            Recommender systems,
            Natural language processing,
            Predictive maintenance,
            Customer engagement,
            Computer vision
        },
        ytick=data, % Use the data for y-ticks
        nodes near coords,
        nodes near coords align={anchor=west}, % Add the percentage labels near the bars% Align the labels horizontally
        point meta=explicit symbolic % The meta data is explicitly given as symbolic text
    ]
    \addplot [draw=blue,fill=blue!30 ] coordinates {
        (51,Computer vision)[51\%]
        (47,Customer engagement)[47\%]
        (43,Predictive maintenance)[43\%]
        (38,Natural language processing)[38\%]
        (34,Recommender systems)[34\%]
        (31,Intelligent search)[31\%]
        (28,Document processing)[28\%]
        (26,Sales and marketing analytics)[26\%]
        (23,Facial recognition)[23\%]
        (18,Image recognition)[18\%]
        (13,Speech recognition)[13\%]};
    \end{axis}
\end{tikzpicture}
\caption[Gängigste Verwendungszwecke von KI in Unternehmen]{Gängigsten Verwendungszwecke von KI in Unternehmen\footnotemark} % Caption for the figure
    \label{fig:ai_tech_distribution} % Label for referencing the figure
\end{figure}
\footnotetext{Entnommen aus: \cite{rackspace_most_2023}}

Die Verarbeitung von gescannten Dokumenten, insbesondere von Rechnungen ist ein integraler Bestandteil der Dokumentenverarbeitung. Der Einsatz von \ac{OCR} ermöglicht die Digitalisierung und elektronische Weiterverarbeitung dieser Dokumente. Jeoch ist die Zuordnung der erkannten Zeichenketten zu interpretierbaren Metadaten oft von manueller Nachbearbeitung oder von regulären Ausdrücken abhängig. Eine weitere Hürde, welche sich bei der Extraktion von Informationen aus Rechnungen zeigt, sind die sehr heterogenen Vorlagen und Layouts, welche in der Verarbeitung zu Ungenauigkeiten führen kann. Die OCR kann hier nicht mehr sicher die erkannten Werte den entsprechenden Labeln zuordnen.\footcites[Vgl.][S.1]{rahal_information_2018} Die DMS-Software \ac{ELO} von ELO verarbeitet Rechnungen zurzeit mittels OCR. Die Anwendung von Transformer-Modellen, im Bereich des \ac{VDU} ermöglicht eine direkte kontextbasierte Weiterverarbeitung der Rechnungen.\footcites[Vgl.][S.1]{kim_ocr-free_2021}

Diese Arbeit untersucht neue Transformer-Modelle im Bereich des VDU. Es wird analysiert, wie diese Modelle, die ihren Ursprng in der natürlichen Sprachverarbeitung haben, auf die Interpretation visueller (layoutbasierter) und textueller Elemente in Dokumenten angewandt werden können. Weiterhin wird ein OCR-freier \ac{Donut} untersucht, um die Limitierungen von OCR bezüglich Laufzeit und Fehleranfälligkeit zu überwinden.\footcites[Vgl.][S.1]{kim_ocr-free_2021} Diese Arbeit umfasst die Bereitstellung einer Pipeline, welche vom Modelltraining des Donuts bis zur Endanwendung im Unternehmen ELO Digital Office GmbH reicht. 
Das Ziel dieser Arbeit ist es, die Anwendbarkeit von Transformer-Modellen im Bereich des \ac{VDU} zu untersuchen und die Leistungsfähigkeit der Pipeline zu evaluieren um folgende Forschungsfrage zu beantworten:
\begin{center}
    \emph{Kann die Anwendung von Document Understanding Transformer Modellen die Effizienz und Genauigkeit der Rechnungsverarbeitung im Vergleich zu bestehenden OCR-basierten Modellen steigern?}
\end{center} 
Um diese Frage zu beantworten wird anhand eines Benchmarks (detaillierte Beschreibung des Aufbaus folgt in Kapitel 3) die Leistungsfähigkeit der Donut-basierten Pipeline mittels diverser Metriken bemessen und zur Leistungsfähigkeit von OCR-basierten Modellen verglichen. Daher ist die Arbeit folgedermaßen strukturiert: \\Zunächst werden in Kapitel 2 die Grundlagen der Dokumentenverarbeitung und der Transformer-Modelle erläutert. In Kapitel 3 werden die zu untersuchenden Modelle ausgewählt. Des weiteren werden der experimentelle Ansatz und der Aufbau der Testumgebung und Datensätze beschrieben. In Kapitel 4 wird die Entwicklung der Pipeline und die Implementierung des Donut dargestellt. Kapitel 5 präsentiert die Ergebnisse des Experiments. Kapitel 6 diskutiert die Ergebnisse und zieht Schlussfolgerungen. Eine Zusammenfassung und ein Ausblick auf zukünftige Arbeiten schließen die Arbeit in Kapitel 7 ab. Mit dieser Strukturierung als Ausgangspunkt, folgt nun die detaillierte Betrachtung der Grundlagen der Dokumentenverarbeitung und der Transformer-Modelle in Kapitel 2.


%%% Ende des eigentlichen Inhalts %%%

% chapter  (end)

%%% Quellenverzeichnisse (keine Anpassung nötig) %%%
\clearpage
\literaturverzeichnis
%%% Ende Quellenverzeichnisse %%%

\clearpage
\clearpage

\thispagestyle{empty}

{\LARGE\textsf{\textbf{Erklärung zur Verwendung generativer KI-Systeme}}\bigskip}

Bei der Erstellung der eingereichten Arbeit habe ich die nachfolgend aufgeführten auf künstlicher Intelligenz (KI) basierten Systeme benutzt:
% HIER EDITIEREN: (Immer mit \item einen neuen Antrag anführen
\begin{enumerate}
  \item
\end{enumerate}

Ich erkläre, dass ich
\begin{itemize}
  \item mich aktiv über die Leistungsfähigkeit und Beschränkungen der oben genannten KI-Systeme informiert habe, \footnote{U. a. gilt es hierbei zu beachten, dass an KI weitergegebene Inhalte ggf. als Trainingsdaten genutzt und wiederverwendet werden. Dies ist insb. für betriebliche Aspekte als kritisch einzustufen.}
  \item die aus den oben angegebenen KI-Systemen direkt oder sinngemäß übernommenen Passagen gekennzeichnet habe, \footnote{In der Fußnote Ihrer Arbeit geben Sie die KI als Quelle an, z.B.: Erzeugt durch Microsoft Copilot am dd.mm.yyyy. Oder: Entnommen aus einem Dialog mit Perplexity vom dd.mm.yyyy. Oder Absatz 2.3 wurde durch ChatGPT sprachlich geglättet.}
  \item überprüft habe, dass die mithilfe der oben genannten KI-Systeme generierten und von mir übernommenen Inhalte faktisch richtig sind, \footnote{Beispiele hierfür sind u.a. die folgenden Arbeitsschritte: Generierung von Ideen, Konzeption der Arbeit, Literatursuche, Literaturanalyse, Literaturverwaltung, Auswahl von Methoden, Datensammlung, Datenanalyse, Generierung von Programmcodes.}
  \item mir bewusst bin, dass ich als Autorin bzw. Autor dieser Arbeit die Verantwortung für die in ihr gemachten Angaben und Aussagen trage.
\end{itemize}

\vspace{1cm}

Die oben genannten KI-Systeme habe ich wie im Folgenden dargestellt eingesetzt:
\begin{center}
  \begin{tabular}{| p{0.25\linewidth} | p{0.2\linewidth} | p{0.5\linewidth} |}
    \hline
    \textbf{Arbeitsschritt in der wissenschaftlichen Arbeit} & \textbf{Eingesetzte(s) KI-System(e)} & \textbf{Beschreibung der Verwendungsweise} \\
    \hline
    % HIER EDITIEREN: (Nächste Zeile beliebig oft kopieren)
                                                             &                                      &                                            \\
    \hline
  \end{tabular}
\end{center}



%%% Erklärung (keine Anpassungen nötig) %%%
% steht ganz am Ende des Dokuments
\cleardoublepage
\clearpage

\thispagestyle{empty}

{\LARGE\textsf{\textbf{Erklärung}}\bigskip}

% \typMeinerArbeit und \themaMeinerArbeit werden in deckblatt.tex definiert
Ich versichere hiermit, dass ich die vorliegende Arbeit mit dem Thema: \emph{\themaMeinerArbeit} selbstständig verfasst und keine anderen als die angegebenen Quellen und Hilfsmittel benutzt habe.
Ich versichere zudem, dass die eingereichte elektronische Fassung mit der gedruckten Fassung übereinstimmt.

\vspace{3cm}

\begin{center}
  \begin{tabular}{ccc}
    (Ort, Datum) & \hspace{0.3\linewidth} & (Unterschrift)
  \end{tabular}
\end{center}
\end{document}
