%% Definitionen für Kopf- und Fußzeile auf normalen Seiten
\defpagestyle{kopfzeile}
{% Kopfdefinition
  (\textwidth,0pt)    % Länge der oberen Linie,Dicke der oberen Linie       
  {} % Definition für linke Seiten im doppelseitigen Layout
    {} % Definition für rechte Seiten im doppelseitigen Layout      
    {  % Definition für Seiten im einseitigen Layout
      \makebox[0pt][l]{\rightmark}% 
      \makebox[\linewidth]{}% 
    }
  (\textwidth, 0.4pt) % Untere Linienlänge, Untere Liniendicke
}
{% Fußdefinition
  (\textwidth,0pt)    % Obere Linienlänge, Obere Liniendicke
  {} % Definition für linke Seiten im doppelseitigen Layout
    {} % Definition für rechte Seiten im doppelseitigen Layout
    {  % Definition für Seiten im einseitigen Layout
      \makebox[\linewidth]{}%
      \makebox[0pt][r]{\pagemark}%
    }
  (\textwidth, 0pt)   % Länge der unteren Linie,Dicke der unteren Linie
}

%% Definitionen für Kopf- und Fußzeile auf ersten Seiten eines Kapitels
\defpagestyle{kapitelkopfzeile}
{% Kopfdefinition
  (\textwidth,0pt)    % Länge der oberen Linie,Dicke der oberen Linie       
  {} % Definition für linke Seiten im doppelseitigen Layout
    {} % Definition für rechte Seiten im doppelseitigen Layout      
    {}  % Definition für Seiten im einseitigen Layout
  (\textwidth, 0pt) % Untere Linienlänge, Untere Liniendicke
}
{% Fußdefinition
  (\textwidth,0pt)    % Obere Linienlänge, Obere Liniendicke
  {} % Definition für linke Seiten im doppelseitigen Layout
    {} % Definition für rechte Seiten im doppelseitigen Layout
    {  % Definition für Seiten im einseitigen Layout
      \makebox[\linewidth]{}%
      \makebox[0pt][r]{\pagemark}%
    }
  (\textwidth, 0pt)   % Länge der unteren Linie,Dicke der unteren Linie
}

%% Definitionen für Kopf- und Fußzeile im Anhang und bei Quellenverzeichnisse
\newcommand{\spezialkopfzeileBezeichnung}{}
\defpagestyle{spezialkopfzeile}
{% Kopfdefinition
  (\textwidth,0pt)    % Länge der oberen Linie,Dicke der oberen Linie       
  {} % Definition für linke Seiten im doppelseitigen Layout
    {} % Definition für rechte Seiten im doppelseitigen Layout      
    {  % Definition für Seiten im einseitigen Layout
      \makebox[0pt][l]{\spezialkopfzeileBezeichnung}% 
      \makebox[\linewidth]{}% 
    }
  (\textwidth, 0.4pt) % Untere Linienlänge, Untere Liniendicke
}
{% Fußdefinition
  (\textwidth,0pt)    % Obere Linienlänge, Obere Liniendicke
  {} % Definition für linke Seiten im doppelseitigen Layout
    {} % Definition für rechte Seiten im doppelseitigen Layout
    {  % Definition für Seiten im einseitigen Layout
      \makebox[\linewidth]{}%
      \makebox[0pt][r]{\pagemark}%
    }
  (\textwidth, 0pt)   % Länge der unteren Linie,Dicke der unteren Linie
}

\newcommand\spezialkopfzeile[1]{%
  \renewcommand\spezialkopfzeileBezeichnung{#1}
  \pagestyle{spezialkopfzeile}
}

% Standard-Pagestyle auswählen
\pagestyle{kopfzeile}

% keine Kopfzeile anzeigen auf Seiten, auf denen ein 
% Kapitel beginnt oder das Inhalts-/Abbildungs-/Tabellenverzeichnis steht 
\renewcommand{\chapterpagestyle}{kapitelkopfzeile}
\tocloftpagestyle{kapitelkopfzeile}

